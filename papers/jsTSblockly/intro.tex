\section{Introduction}

Blockly~\cite{Blocky2015} (\url{github.com/google/blockly}) 
is a JavaScript framework for creating visual block-based editing environments, 
inspired by MIT’s Scratch programming environment~\cite{ScratchCACM2009}.  Visual blocks represent structured 
control-flow statements such as while loops and if-then-else conditionals, as well as program expressions,
values and variables. 
Blocks also represent function calls to domain-specific APIS. Blocky supports the creation of \emph{custom 
blocks} for domain-specific APIs. Blockly can be compiled to a variety of languages, the main target 
being JavaScript (as Blockly itself is written in JavaScript and hosted in a web browser).

TypeScript (\url{www.typescriptlang.org}) is superset of the JavaScript language that is gradually typed, 
meaning that types may 
optionally be added to JavaScript code to provide for a more productive programming and debugging experience. 
We describe a mapping from TypeScript annotated JavaScript APIs to Blockly that greatly simplifies 
the process of making existing JavaScript frameworks and libraries available via Blockly.

Our goal is not to have a mapping of every TypeScript programming construct into Blockly 
(i.e., a one-to-one mapping of the TypeScript abstract syntax tree nodes to blocks).
Rather, it is support common programming paradigms with a simple mapping
from TypeScript to blocks and back.  This mapping may, in fact, be many-to-one, 
abstracting over multiple TypeScript elements to provide a single block abstraction.
This supports Blockly's goal to provide a simplified programming experience 
with higher-level abstractions.

\subsection{Example}

Here's an example of a function that takes two parameters (an enumeration and
an event handler - a lambda function):
\begin{lstlisting}
enum Button { A, B };
function onButton(button:  Button, 
                  handler: () => void): void { }
\end{lstlisting}
The function is explicitly typed using TypeScript's support for
type annotations: each parameter has a type, which follows the colon;
furthermore, the return type of the function also is specified as void
via annotation. 

This function gives rise to a 
a block named ``onButton'', shown in Figure~\ref{fig:onButton}
(later sections described how attributes in the comment preceding
the function allow the look and feel of the block to be customized).
The function definition and its types define the structure of the
block:
\begin{itemize}
\item  a \emph{value} input corresponding to the parameter
\emph{button} which allows selection of one of two 
possible parameter values, either ``A'' or ``B'', corresponding to the
the two possible values for the enumeration \emph{Button};
while an enumeration is realized as a JavaScript number, we
generate a Blockly field selector for this parameter that allows
the user to choose either ``A'' or ``B''.
\item the block has a \emph{statement} input corresponding to the
statements that comprise the body of the event handler \emph{handler},
which has the function type \emph{() => void}. 
\end{itemize}
So, we see that the block allows user to instantiate the code fully without
the necessity to explicitly create a lambda function.

\subsection{Overview}



% Nonetheless, there are features of the TypeScript language that one usually
% finds lacking in Blockly instantations, such as explicit use of lexical scoping,


% \begin{itemize}
%   \item lexical scoping (via let);
%   \item functions as values (also nested functions -> closures);
%   \item classes with single inheritance, methods, objects
%   \item arrays (lists);
%   \item object destructuring
% \end{itemize}
