\documentclass[sigplan,10pt]{acmart}
%\settopmatter{printfolios=false,printccs=false,printacmref=false}
\usepackage{graphicx}
\usepackage{listings}
\usepackage{enumitem}
\newcommand{\MC}{MakeCode\ }
\newcommand{\MCN}{MakeCode}
\newcommand{\CO}{CODAL\ }
\newcommand{\CON}{CODAL}
\newcommand{\COLN}{codal}
\newcommand{\UF}{UF2\ }
\newcommand{\UFN}{UF2}
\newcommand{\flameon}[1]{\emph{#1}}
\def\dbhref#1#2{URL}
%\def\dbhref#1#2{\href{#1}{#2}}
\def\dburl#1{URL}
%\def\dburl#1{\url{#1}}

\setlist[itemize]{leftmargin=*}
\lstset{ %
language=C++,                % choose the language of the code
basicstyle=\footnotesize,       % the size of the fonts that are used for the code
numbers=left,                   % where to put the line-numbers
numberstyle=\footnotesize,      % the size of the fonts that are used for the line-numbers
stepnumber=1,                   % the step between two line-numbers. If it is 1 each line will be numbered
numbersep=5pt,                  % how far the line-numbers are from the code
backgroundcolor=\color{white},  % choose the background color. You must add \usepackage{color}
showspaces=false,               % show spaces adding particular underscores
showstringspaces=false,         % underline spaces within strings
showtabs=false,                 % show tabs within strings adding particular underscores
frame=single,           % adds a frame around the code
tabsize=2,          % sets default tabsize to 2 spaces
captionpos=b,           % sets the caption-position to bottom
breaklines=true,        % sets automatic line breaking
breakatwhitespace=false,    % sets if automatic breaks should only happen at whitespace
escapeinside={\%*}{*)}          % if you want to add a comment within your code
}

%% For double-blind review submission, w/ CCS and ACM Reference
%\documentclass[sigplan,10pt,review,anonymous]{acmart}\settopmatter{printfolios=true}
%% For single-blind review submission, w/o CCS and ACM Reference (max submission space)
%\documentclass[sigplan,10pt,review]{acmart}\settopmatter{printfolios=true,printccs=false,printacmref=false}
%% For single-blind review submission, w/ CCS and ACM Reference
%\documentclass[sigplan,10pt,review]{acmart}\settopmatter{printfolios=true}
%% For final camera-ready submission, w/ required CCS and ACM Reference
%\documentclass[sigplan,10pt]{acmart}\settopmatter{}


%% Conference information
%% Supplied to authors by publisher for camera-ready submission;
%% use defaults for review submission.
% \acmConference[PL'17]{ACM SIGPLAN Conference on Programming Languages}{January 01--03, 2017}{New York, NY, USA}
% \acmYear{2017}
% \acmISBN{} % \acmISBN{978-x-xxxx-xxxx-x/YY/MM}
% \acmDOI{} % \acmDOI{10.1145/nnnnnnn.nnnnnnn}
\startPage{1}

%% Copyright information
%% Supplied to authors (based on authors' rights management selection;
%% see authors.acm.org) by publisher for camera-ready submission;
%% use 'none' for review submission.
\setcopyright{none}
%\setcopyright{acmcopyright}
%\setcopyright{acmlicensed}
%\setcopyright{rightsretained}
%\copyrightyear{2017}           %% If different from \acmYear

%% Bibliography style
\bibliographystyle{ACM-Reference-Format}
%% Citation style
%\citestyle{acmauthoryear}  %% For author/year citations
%\citestyle{acmnumeric}     %% For numeric citations
%\setcitestyle{nosort}      %% With 'acmnumeric', to disable automatic
                            %% sorting of references within a single citation;
                            %% e.g., \cite{Smith99,Carpenter05,Baker12}
                            %% rendered as [14,5,2] rather than [2,5,14].
%\setcitesyle{nocompress}   %% With 'acmnumeric', to disable automatic
                            %% compression of sequential references within a
                            %% single citation;
                            %% e.g., \cite{Baker12,Baker14,Baker16}
                            %% rendered as [2,3,4] rather than [2-4].


%%%%%%%%%%%%%%%%%%%%%%%%%%%%%%%%%%%%%%%%%%%%%%%%%%%%%%%%%%%%%%%%%%%%%%
%% Note: Authors migrating a paper from traditional SIGPLAN
%% proceedings format to PACMPL format must update the
%% '\documentclass' and topmatter commands above; see
%% 'acmart-pacmpl-template.tex'.
%%%%%%%%%%%%%%%%%%%%%%%%%%%%%%%%%%%%%%%%%%%%%%%%%%%%%%%%%%%%%%%%%%%%%%


%% Some recommended packages.
\usepackage{booktabs}   %% For formal tables:
                        %% http://ctan.org/pkg/booktabs
\usepackage{subcaption} %% For complex figures with subfigures/subcaptions
                        %% http://ctan.org/pkg/subcaption

\usepackage{courier}

\begin{document}

%% Title information
\title{TypeScript: From JavaScript \\ to Blockly and Back}         %% [Short Title] is optional;
                                        %% when present, will be used in
                                        %% header instead of Full Title.
\subtitle{Microsoft MakeCode Team}                     %% \subtitle is optional


%% Author information
%% Contents and number of authors suppressed with 'anonymous'.
%% Each author should be introduced by \author, followed by
%% \authornote (optional), \orcid (optional), \affiliation, and
%% \email.
%% An author may have multiple affiliations and/or emails; repeat the
%% appropriate command.
%% Many elements are not rendered, but should be provided for metadata
%% extraction tools.

% %% Author with single affiliation.
% \author{James Devine}
% \affiliation{
%   \institution{Lancaster University, UK}            %% \institution is required
% }
% \email{james@devine.eu}  
% \author{Joe Finney}
% \affiliation{
%   \institution{Lancaster University, UK}            %% \institution is required
% }
% \email{j.finney@lancaster.ac.uk} 
% \author{Micha\l Moskal}
% \affiliation{
%   \institution{Microsoft, USA}            %% \institution is required
% }
% \email{mmoskal@microsoft.com} 
% \author{Peli de Halleux}
% \affiliation{
%   \institution{Microsoft, USA}            %% \institution is required
% }
% \email{jhalleux@microsoft.com} 
% \author{Thomas Ball}
% \affiliation{
%   \institution{Microsoft, USA}            %% \institution is required
% }
% \email{tball@microsoft.com} 
% \author{Steve Hodges}
% \affiliation{
%   \institution{Microsoft, UK}            %% \institution is required
% }
% \email{shodges@microsoft.com} 

%% Author with two affiliations and emails.
% \author{First2 Last2}
% \authornote{with author2 note}          %% \authornote is optional;
%                                         %% can be repeated if necessary
% \orcid{nnnn-nnnn-nnnn-nnnn}             %% \orcid is optional
% \affiliation{
%   \position{Position2a}
%   \department{Department2a}             %% \department is recommended
%   \institution{Institution2a}           %% \institution is required
%   \streetaddress{Street2a Address2a}
%   \city{City2a}
%   \state{State2a}
%   \postcode{Post-Code2a}
%   \country{Country2a}                   %% \country is recommended
% }
% \email{first2.last2@inst2a.com}         %% \email is recommended


%% Abstract
%% Note: \begin{abstract}...\end{abstract} environment must come
%% before \maketitle command
% \begin{abstract}
    Microcontrollers, the low-power low-cost workhorses of embedded systems, see increasing use in making, education, and the Internet of Things.

    Traditionally, microcontrollers require additional software to load programs, and programs are written in low-level languages like C and C++. Given the rise of less traditional application domains, there is now a need to simplify the way we program them.

    We present a new programming platform that makes it easy for \emph{anyone} to program microcontrollers from \emph{anywhere}, using a computer with a modern web browser and a USB port (no native applications, C/C++ compilers, or drivers required).

    We describe how the platform has been architected to bridge the worlds of the Web, and a wide range of microcontrollers (\emph{anything}) via language, compiler, and runtime innovations.

    We evaluate the performance of the platform on devices ranging from small 8-bit processors with just 2kB of RAM, to more endowed processors.
\end{abstract}


\begin{abstract}
  There are many JavaScript libraries for building solutions for a 
  wide range of problems, but it's not easy for novices to harness their power.  
  The Blockly framework allows the creation of domain-specific 
  programming environments for novices that greatly reduces the potential
  for syntax and semantic errors.  We show how the TypeScript
  programming language, a gradually typed superset of JavaScript, can be used
  to bridge the jap between JavaScript and Blockly. In particular, we define
  a mapping from the type system of TypeScript into Blockly that makes it simple
  to create a domain-specific Blockly editor for a JavaScript library via a 
  TypeScript declaration file. 
\end{abstract}


%% Keywords
%% comma separated list
%\keywords{keyword1, keyword2, keyword3}  %% \keywords are mandatory in final camera-ready submission


%% \maketitle
%% Note: \maketitle command must come after title commands, author
%% commands, abstract environment, Computing Classification System
%% environment and commands, and keywords command.
\maketitle

\section{Introduction}

\section{Example}

All exported functions with a block attribute will be available in the Block Editor.

\begin{lstlisting}
//% block
export function showNumber(v: number, interval: number = 150): void
{ }
\end{lstlisting}
If you need more control over the appearance of the block, you can specify the 
blockId and block parameters:
\begin{lstlisting}
//% blockId=device_show_number
//% block="show|number %v"
export function showNumber(v: number, interval: number = 150): void
{ }
\end{lstlisting}

% blockId is a constant, unique id for the block. This id is serialized in block code so changing it will break your users.
% block contains the syntax to build the block structure (more below).
% Other optional attributes can also be used:

% blockExternalInputs= forces External Inputs rendering
% advanced=true causes this block to be placed under the parent category’s “More…” subcategory. Useful for hiding advanced or rarely-used blocks by default
% Block syntax

\section{Blockly Overview}

Unsuprisingly, the core abstract of Blockly is the \emph{Block},
which is used to represent statements, expressions, values and variables.  
Blocks have \emph{connectors} that allow them to be sequenced
horizontally or vertically in space.  In the default Blockly
layout, vertically sequenced blocks
generally represent program statements, while horizontally
sequenced blocks generally represent program expressions/values.

\emph{[Blockly supports nesting of blocks, so leads naturally to representing an
abstract syntax tree]}

While Blockly does attach a loose meaning to blocks,
their actual semantics given to blocks is entirely up to the developer.
In our work, we use the TypeScript language (and its type system) to 
give a more precise meaning to blocks. 

\subsection{Block connections}

Block connectors, inputs and output:
\begin{itemize}
% A block with a previous connector cannot have an output connector, and vice versa. 
% The term statement block refers to a block with no value output. 

% nextStatement and previousStatement connections can be typed
% but this feature is not utilized by standard blocks.

\item \emph{previous and next connectors}, both optional, allow a block to be vertically sequenced - Blockly
  enforces that a block with a previous connector cannnot have an output connector; according
  to Blockly documentation, ``a statement block will usually have both a previous connection and 
  a next connection'';

\item a block may have a single \emph{output connector}, 
      which appears as a male jigsaw connector on a block's left-side; according to 
      Blockly documentation, ``Blocks with an output are usually called value blocks'';

\item a block may have multiple \emph{inputs}, which can appears as holes in a block
      or female jigsaw connectors on a block's right-side, about which we'll say more below. 
\end{itemize}

Block inputs come in three basic forms:
\begin{enumerate}
  \item \emph{fields}, which represent terminals (constants, literals, variables);
  \item \emph{value inputs} (receives value from output block) - value inputs
      and value blocks allow one to create expression trees;
  \item \emph{statement inputs}, allow one to create statement trees;
\end{enumerate}


\subsection{Blockly types}

\subsection{Blockly conventions (JavaScript)}
\begin{itemize}
  \item no variable declarations (all variables global, except for);
  \item dynamic typing;
  \item expressions can't be put where statement expected 
  - use an assignment to a dummy variable to make expression into statement);
  \item loops and local scope
\end{itemize}

\section{Primitive Values and Enumerations}

% The following types are supported in function signatures that are meant to be exported:

% boolean
% number (TypeScript)
% string (TypeScript) 
% enums (see below)

\section{API Function Calls}

API represented by
\begin{itemize}
  \item namespaces to organize set of related functions, 
       where namespaces map to Toolbox categories in Blockly;
       In particular, each top-level namespace is used to populate a category 
       in the toolbox. The name of the namespaces is capitalized for the toolbox. 

  \item void return type for a function means a call to the function is allowable in
      a statement context; non-void means expression context for call
  \item functions return a single value;
  \item parameters: optional parameters with default value; rest; 
\end{itemize}

% Block syntax

% The block attribute specifies how the parameters of the function will be organized to create the block.

% block = field, { '|' field }
% field := string
%     | string `%` parameter [ `=` type ]
% parameter = string
% type = string
% each field is mapped to a field in the block editor
% the function parameter are mapped in order to %parameter argument. 

% The loader automatically builds a mapping between the block field names and the function names.
% the block will automatically switch to external inputs when enough parameters are detected
% A block type =type can be specified optionally for each parameter. It will be used to populate the shadow type.
% Supported types

\section{Objects and Object Destructuring}
API represented by
\begin{itemize}
  \item set of classes, with
  \item constructors
  \item methods/fields
\end{itemize}

\section{Event Handlers and Callbacks}

% \item event-handlers via a callback function (callbacks only have parameters with primitive values);

A function $f$ that has an argument $g$ of function type (in last position) will have
that function argument (callback) converted into a statement input of $B(f)$.

If the callback $g$ has parameters, then
the best way to map that pattern to the blocks is by modifying
$g$ to have a single parameter with a class type that has the
various parameters as fields.
For example:

% export class ArgumentClass {
%     argumentA: number;
%     argumentB: string;
% }

% //% mutate=objectdestructuring
% //% mutateText="My Arguments"
% //% mutateDefaults="argumentA;argumentA,argumentB"
% // ...
% export function addSomeEventHandler((a: ArgumentClass) => void) { };

% In the above example, setting mutate=objectdestructuring will cause this API 
% to use Blockly “mutators” to let users change what parameters appear in the blocks. 
% Each parameter will be given an optional variable field in the block that defines a 
% variable that can be used within the callback. The variable fields compile to object
% destructuring in the TypeScript code. For example:

% addSomeEventHandler(({argumentA, argumentB}) => {

% })

% For an example of this pattern in action, see the radio.onDataPacketReceived block in the microbit target.

% In some cases it can be useful to change the runtime behavior of the API based on the properties 
% selected by the user. To enable that behavior, create an enum with entries that have the same names 
% as the argument object’s properties and add an extra parameter taking in an enum array to the API. 
% For example:

% export class ArgumentClass {
%     argumentA: number;
%     argumentB: string;
% }

% enum ArgNames {
%     argumentA,
%     argumentB
% }

% //% mutate=objectdestructuring
% //% mutateText="My Arguments"
% //% mutateDefaults="argumentA;argumentA,argumentB"
% //% mutatePropertyEnum="argNames"
% // ...
% export function addSomeEventHandler(args: ArgNames[], (a: ArgumentClass) => void) { };
% Note the mutatePropertyEnum attribute added to the comment annotations. The block for this API will look the same as the previous example but the compiled code will also include the arguments passed:

% addSomeEventHandler([ArgNames.argumentA, ArgNames.argumentB], ({argumentA, argumentB}) => {

% })
% The other attributes related to object destructuring mutators include:

% mutateText - defines the text that appears in the top block of the Blockly mutator dialog (the dialog that appears when you click the blue gear)
% mutateDefaults - defines the versions of this block that should appear in the toolbox. Block definitions are separated by semicolons and property names should be separated by commas




\section{Collections}

\section{Monads}

\section{Lexical Scoping}

\section{Attributes}

%% Bibliography
%\bibliography{paper}

%% Appendix
%\appendix
%\section{Appendix: Static TypeScript Subtype Relation}

In STS, $S$ is a subtype of a type $T$ if one of the following is true:
\begin{itemize}
\item $S$ and $T$ are identical types;
\item $S$ is the Undefined type;
\item $S$ is the Null type and $T$ is not the Undefined type;
\item $S$ is an enum type and $T$ is the primitive type Number;
\item $S$ is a string literal type and $T$ is the primitive type String;
\item $S$ and $T$ are class types and all the following are true:
\begin{itemize}
  \item $S$ is derived from $T$ (via \emph{extends} clauses);
  \item checkProps($S$,$T$) holds;
\end{itemize}
\item $S$ is a class/record type and $T$ is a record type and
\begin{itemize}
  \item checkProps($S$,$T$) holds;
\end{itemize}
\item $S$ and $T$ are function types such that all the following hold:
\begin{itemize}
  \item $S$ has at least as many parameters as $T$;
  \item each parameter type in $T$ is a subtype of the corresponding parameter type in $S$;
  \item the result type of $T$ is Void, or the result type of $S$ is a subtype of that of $T$;
\end{itemize}
\item $S$ and $T$ are array types and all the following hold:
\begin{itemize}
\item $T$ has a numeric index signature with element type $U$, 
    and $S$ has a numeric index signature with element type $V$
    such that $V$ is a subtype of $U$;
\item checkProps($S$,$T$) holds.
\end{itemize}
\end{itemize}

% https://www.earthli.com/news/view_article.php?id=3391

Given types $S$ and $T$, checkProps($S$,$T$) holds if for each property $N$ in $T$, 
$S$ has a property $M$ where all of the following are true:
\begin{itemize}
\item $M$ and $N$ have the same name;
\item the type of $M$ is a subtype of the type of $N$;
\item $M$ and $N$ are both public, or $M$ and $N$ are both 
      private (protected) and originate in the same declaration, 
      or $N$ is protected and $S$ is a class derived from class $T$
\end{itemize}
%\section{USB Flashing Format}
\label{sec:uf2}

The way that code gets from a host computer onto a microcontroller is deeply rooted in 1980's technologies - 
serial wires, obscure protocols, and text based file formats with limited line length. Depending on exact circumstances,
one must install serial USB drivers, select the right port and parameters, and then use a native application to access
the microcontroller. As the advance of maker content in educational curricula continues,
this complicated flashing process presents one of the major obstacles to adoption in schools, 
where installing any software is usually the domain of IT administrators.

One solution would be to rely on emerging standards, like WebUSB and WebBluetooth to transfer a program from the browser 
to the microcontroller. These standards are however still in their infancy, and it may take even longer before they are 
deployed in schools.

Another solution was pioneered by ARM with its DAPLink firmware, where a USB-capable microcontroller presents itself 
to an external computer as a USB pen drive. No special drivers need to be installed, as operating systems support pen
drives out of the box, typically formatted using FAT. The USB pen drive protocol (Mass Storage Class or MSC) is a
block-level protocol (generally 512 bytes) with no concept of files. DAPLink exposes a virtual FAT file system, which
is very small and never changes due to OS writes - it has an informational text file and an HTML file with a redirect
to the online editing environment. Otherwise, the FAT and the root directory table are empty. When the OS tries to
read a block, the DAPLink computes what should be there, based on compiled-in contents of the info and HTML files.
On file system writes, DAPLink detects the Intel HEX format~\cite{IntelHEX}, 
decodes it and flashes the file's contents into the target microcontroller's memory. 

Let's consider the problem that DAPLink must solve: it sees a 512 byte block of data to be written
at a particular block index on the device and must decide if it's part of the file being flashed and, if so, extract
the data and write it to the target microcontroller. Furthermore, when the OS writes a HEX file, DAPLink needs to discard
writes to the FAT or directory table, as well as writes of the meta-data files. It may need to deal with out-of-order writes.
All these details mean that DAPLink is quite complex and sometimes needs to be updated when a new OS release changes the way
in which it handles FAT. This also is the reason that some MSC bootloaders for various chips only support given operating
systems under specific conditions.

The task would be simplified then, if every 512 block of the file being flashed was easy to distinguish from meta-data
or other random files, and easy to act on, independent of other blocks. Intel's HEX file format doesn't give us these properties 
- the 512 byte boundary can be in the middle of a line in the HEX file, and even if we have an entire line, every line only
contains the last 16 bits of the address where to flash, with the upper 16 supplied only when they change.

For these reasons, we designed the USB Flashing Format (UF2). It consists of 512 byte blocks, where each block contains:
\begin{itemize}
\item magic numbers at the beginning and end (to heuristically distinguish it from any other data the OS writes);
\item the address in the target chip flash memory where the payload should be written;
\item the payload data (up to 476 bytes)
\end{itemize}

% A UF2 file consists of 512 byte blocks. Each block starts with a 32 byte header, followed by data, and a final magic number. All fields, except for data, are 32 bit unsigned little endian integers.

% Offset	Size	Value
% 0	4	First magic number, 0x0A324655 ("UF2\n")
% 4	4	Second magic number, 0x9E5D5157
% 8	4	Flags
% 12	4	Address in flash where the data should be written
% 16	4	Number of bytes used in data (often 256)
% 20	4	Sequential block number; starts at 0
% 24	4	Total number of blocks in file
% 28	4	File size or reserved (write as zero)
% 32	476	Data, padded with zeros
% 508	4	Final magic number, 0x0AB16F30

The file format is designed specifically to deal with the following problems:

\begin{itemize}
\item operating system (OS) writing blocks in different order than occurs in a file;
\item OS writing blocks multiple times;
\item OS writing data that is not a UF2 block;
\item OS writing first/final part of a block, possibly for metadata detection or search indexing;
\end{itemize}

The magic number at the end helps to mitigate partial block writes.
Second and final magic numbers were randomly selected, 
except for the last byte of final magic number, which was forced to be `\\n' (0xA). 
Together with the first magic number being ``UF2\\n'' this makes it easy to identify UF2 blocks in a text editor.


The header is padded to 32 bytes, as hex editors commonly use 16 or 32 bytes as line length. 
This way, the data payload is aligned to line start.

32 bit integers are used for all fields so that large flash sizes can be supported in future, as well as for simplicity.
Little endian is used, as most microcontrollers are little endian. 8 bit microcontrollers can choose to just use the
first 16 bits of various header fields.

The total number of blocks in the file and the sequential block number make it easy 
bootloaders to detect that all blocks have been transferred. It requires one bit of 
memory per block (eg., on SAMD21G18A it's 128 bytes). Alternatively, the bootloader might
ignore that and just implement a reset after say 1 second break in incoming UF2 blocks.

The only file system assumption we make is that blocks of file are aligned with blocks on the hard drive. 
It's likely true of many file systems besides FAT. We also assume that USB MSC device reports its block 
size to be a multiple of 512 bytes. In the wild these devices always almost report exactly 512, and some 
operating systems do not support other values.

\subsection{Overheads}

Target chips usually can only write their flash in chunks larger than page size. On the SAMD21, pages are 64 bytes 
but need to be erased four pages at once, so the effective page size is 256 bytes. Therefore, UF2 for SAMD21 uses 
only 256 out of the 476 byte payload, so every block can written to flash straight away. This is still more 
efficient than HEX (which stores every byte as two ASCII characters and adds 20-30\% overhead). It also doesn’t 
matter - the file is transferred at USB full speed (around 1MB per second), and so the limiting factor is writing 
to flash.

Each block also can contain an optional total number of blocks in the file and the current block number. 
This lets the bootloader detect the end of the transfer (by keeping bitmap of all written blocks, 
as we assume blocks can be written more than once).




\end{document}
