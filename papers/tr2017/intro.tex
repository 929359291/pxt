\section{Introduction}
\label{sec:intro}

Over the last decade, microcontrollers, the workhorses of embedded systems, have become 
central to efforts in making and education. For example, the Arduino project (\url{www.arduino.cc}), 
started in 2003, created a printed circuit board (the Uno) based on the 8-bit Atmel 
AVR microcontroller unit that makes most of its I/O pins available via headers. Hardware modules 
(shields) may be connected to the main board to extend its capability. 
The Arduino ecosystem, based on an open hardware design, has grown tremendously in the past 15 years, 
with the support of companies such as Adafruit Industries (\url{www.adafruit.com}) and 
Sparkfun Electronics (\url{www.sparkfun.com}), to name a few.

In contrast, what has not changed much is the way microcontrollers are programmed,
which is with the C/C++ programming language (as well assembly).   
This is not a huge surprise, given the low-level nature of microcontroller programming, 
where direct access to the hardware is the order of the day. There generally is no operating 
system running on such boards, as they have very little RAM (2K for the Uno, for example) and 
lack memory protection hardware.  What is more surprising about the Arduino platform is that:
\begin{itemize}
\item it encourages the programmer to use polling to interact with sensors, 
which leads to monolithic sequential programs;
\item its IDE lacks any code ``intellisense'' or common interactive features of modern IDEs;
\item it loads code onto the microcontroller using 1980s era bootloader technology.
\end{itemize}
As a result, it is not simple to get started with Arduino-based systems, of which there are many. 
On the other hand, on the web we find many excellent environments for introductory programming. 
Visual block-based editors such as Scratch (\url{https://scratch.mit.edu/}) and Blockly (\url{https://developers.google.com/blockly/}) 
allow the creation of programs without the possibility of syntax errors. 
HTML, CSS and JavaScript allow a complete programming experience to be delivered as an interactive 
web app, including editing with intellisense, code execution and debugging. (While the Arduino IDE recently 
has been ported to the web, it lacks many of the above features and requires a web connection to a server which runs 
a C/C++ compile tool chain to compile user code.) The programming models associated with these environments are 
event-based, freeing the programmer from the need to poll.

We have created a new programming platform that bridges the worlds of the microcontroller
and the web app. The major goals of the platform are to: (1)
make it simple to program microcontrollers using an interactive web app that also functions offline;
(2) allow a user's compiled program to be easily installed on a microcontroller;
(3) support the addition of new of software/hardware components to a microcontroller.
The platform is defined by four new technologies:
\begin{itemize}
      
\item 
\emph{MakeCode} is a new web app (\url{www.makecode.com}) that supports 
both visual block-based programming and text-based programming using TypeScript (\url{www.typescriptlang.org}), 
a gradually-typed superset of JavaScript. 
The web app also converts between the two program representations. The web app supports in-browser execution 
via a device simulator, as well as compilation to machine code and linking against a 
pre-compiled C++ runtime. No C/C++ compiler is invoked to compile user code. 
The result of compilation is a binary file that is ``downloaded'' from the web app to the user's computer. 

\item \emph{Static TypeScript} is a statically-typed subset of TypeScript for fast execution on low-memory devices 
and a simple model for linking against pre-compiled C++; Static TypeScript also can be used to write safe 
device driver code.

\item \emph{CODAL}, the Component-oriented Device Abstraction Layer, is a new C++ library that maps 
each hardware component of a device to one or more software components that communicate over a message bus and
schedule event handlers to run non-preemptively on fibers. 

\item \emph{USB Flashing Format} (UF2) is a new file format designed for flashing microcontrollers over the Mass Storage
Class (removable USB pen drive) protocol.  This new file format greatly speeds the installation of user programs
and is robust to difference in operating systems.
\end{itemize}
These advances enable beginners to get started programming microcontrollers from any modern web browser, and enable
hardware vendors to innovate and safely add new components to the mix using Static TypeScript, and its
foreign function interface to C++. Once the web app has been loaded, 
all the above functionality works offline (i.e., if the host machine loses its connection 
to the internet).
All of the above components are open source under the MIT/Apache licenses, as detailed below. 

Platform targets can be seen at \url{www.makecode.com}, where the MakeCode web app for a variety of boards is available, 
including the micro:bit (a Nordic nRF51822 microcontroller with Cortex-M0 processor, 16K RAM), Adafruit's Circuit 
Playground Express (CPX: an Atmel SAMD21 microcontroller with Cortex-M0 processor, 16K RAM), and the Arduino Uno 
(Uno: an Atmel ATmega328 microcontroller with AVR processor, 2K RAM). 

We encourage the reader to choose a board and experiment with programming it, using the simulator to explore many 
of each board's features, to appreciate the qualitative aspects of the platform: its simplicity and ease of use.  In this 
paper, we will evaluate quantitative aspects of the platform: 
compilation speed, code size, and runtime performance.  In particular, we evaluate:
\begin{itemize}
\item the compile time of Static TypeScript compile/link of user code (to machine code) with respect 
      to the GCC-based C/C++ toolchain, as well as the size of the resulting executable;
\item the time to load code onto a microcontroller using UF2, compared to standard bootloaders; 
\item The performance of a set of small benchmarks, written in both Static TypeScript and C++,
      compiled with the MakeCode and GCC-based toolchains, as well as the performance of device drivers
      written in Static TypeScript compared to their C++ counterparts.
\end{itemize}
[evaluate with respect to the popular Arduino toolset, for boards with 8-bit (AVR) and 32-bit (Cortex-M0) microcontrollers. 
Summary of evaluation]

The platform's components are open source on GitHub. The MakeCode framework is at \url{https://github.com/microsoft/pxt}.
(PXT is previous codename of MakeCode). 
MakeCode targets for the three previously mentioned boards are at 
\href{https://github.com/microsoft/pxt-microbit}{microsoft/pxt-microbit}, 
\href{https://github.com/microsoft/pxt-adafruit}{microsoft/pxt-adafruit}, and
\href{https://github.com/microsoft/pxt-arduino-uno}{microsoft/pxt-arduino-uno}.
The latter two targets make use of a common set of MakeCode libraries (packages) at
\href{https://github.com/microsoft/pxt-common-packages}{microsoft/pxt-common-packages},  
Many other MakeCode packages, developed by Microsoft and 
hardware partners \emph{ [details later]. A few examples: XYZ.  }

%Lancaster University githubs:
%•	https://github.com/lancaster-university/codal-core
%•	https://github.com/lancaster-university/codal 
%•	micro:bit DAL: https://github.com/lancaster-university/microbit-dal, predecessor of CODAL 
%UF2 githubs
%•	Specification: https://github.com/microsoft/uf2 
%•	SAMD21: https://github.com/microsoft/uf2-samd21
%•	Atmega32UP???

The rest of this paper is organized as follows. Section~\ref{sec:makecode} presents the design and implementation of the MakeCode framework. 
Section~\ref{sec:sts} describes Static TypeScript and section~\ref{sec:codal} presents the CODAL C++ runtime. 
Section~\ref{sec:uf2}: USB Flashing Format;
Section~\ref{sec:evaluate}: Evaluation;
Section~\ref{sec:related}: Related Work;
Section~\ref{sec:conclude}: Conclusion and Future Directions. 
