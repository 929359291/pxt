\section{Related Work}
\label{sec:related}

Related work breaks into four parts: the programming of microcontrollers, 
type systems for JavaScript, operating systems
for microcontrollers, and techniques for code loading.

\subsection{Programming microcontrollers}

\begin{itemize}
\item binary compiled on host computer and loaded; this is the common paradigm for Arduino;
\item command and control via host computer (Firmata, Bluetooth)
\item complete VM on microcontroller (MicroPython: \url{http://micropython.org/})
\end{itemize}

Web-based programming

\begin{itemize}
\item Arduino IDE
\item scratch and scratch extensions (provide tethered mode for microcontrollers)
\item 
\end{itemize}

\subsection{Types for JavaScript}

Type systems with soundness guarantees for JavaScript are numerous. 
Safe TypeScript~\cite{SafeTypeScript15}: distinguishes dynamic from Any type - dynamic means
one of the known static types, where as Any denotes types coming
from raw JavaScript.
StrongScript~\cite{StrongScriptECOOP15}.

Static TypeScript differs from these efforts by outlawing untyped or optionally typed
code.  In this sense, it can be seen to be StrongScript where every variable and 
expression has a concrete (!) type.   As in StrongScript, classes are nominally typed,
which permits a more efficient and traditional property lookup for class instances. 
Static TypeScript goes further by outlawing downcasts (for untagged implementation).

% •	Quite a few others…
% MicroPython

\subsection{Operating systems}

% https://state-machine.com/arduino/ 
% http://tinyos.stanford.edu/tinyos-wiki/index.php/TinyOS_Overview 

% Tethered modes: Kodu for micro:bit, Scratch for micro:bit. The PC is central, the microcontroller is peripheral
% Virtual machines: MicroPython, ev3, littlebits, etc.
% Raspberry Pi: complete operating system

\subsection{Code loading}

% ARM solution