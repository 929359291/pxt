\section{Static TypeScript}
\label{sec:sts}

TypeScript is a typed superset of JavaScript designed to enable JavaScript developers to take advantage of code
completion, static checking and refactoring made possible by types (\url{http://www.typescriptlang.org/}).
As a starting point, every JavaScript program is a TypeScript program.  Types can be added gradually to such programs,
supported by type inference.
In TypeScript, the Any type
represents any JavaScript value with no constraints. Type inference may assign the Any type to expressions for which
no more specific type can be inferred.

While TypeScript provides classes and interfaces with syntax like Java/C\#, their semantics
are quite different, as they are based on JavaScript.  Classes are simply syntactic sugar for creating objects that
have code associated with them, but these objects are JavaScript objects with all their dynamic semantics intact.
%This
%is not surprising, as TypeScript was designed to accommodate the dynamic nature of JavaScript and %programming patterns
%familiar to the JavaScript programmer.

% TOM: DOES UNSOUND MEAN THE USE OF POINTERS?
We approach TypeScript from the viewpoint of the MCU programmer, who is
familiar with the static (though unsound) type systems of C and C++. Our realization is that TypeScript contains a
statically-typed sound subset, Static TypeScript (STS), that closely resembles Java and C\# in its semantics.
STS arises from TypeScript by:
\begin{itemize}
\item \emph{eliminating} the Any type from the type system, as well as JavaScript statements
and expressions that only can be typed with Any;
\item \emph{partitioning} TypeScript's space of object types into functions, records, constructor functions, and arrays, with no casts
    possible between the partitions (and eliminating all other object types) - structural subtyping is used to relate records
    to each other, as in TypeScript;

    % TOM: NOMINALLY? CLARIFICATION REQUIRED?
\item typing of classes \emph{nominally} rather than structurally, as in TypeScript;
\item typing of functions/methods via traditional function subtyping (contravariant
    in the argument type, covariant in the return type)
\item \emph{restricting} casts between records and classes: a class can be cast to a record, but a record cannot be cast to a class.
\end{itemize}
STS is a syntactic subset of TypeScript -- there is no change to the syntax of TypeScript.
In terms of type checking, STS also is a subset of TypeScript. Simply put, given a STS program $P$,
for every type compatibility or subtype relation $R(P)$ of TypeScript, the corresponding relation $R'(P)$ in STS
is a subset of relation $R$. Every STS program is a TypeScript program.

STS guarantees the absence of runtime type errors, including downcasts.
The major class of runtime errors still possible is dereference of null/undefined values.
STS can be supported by a simple runtime model, like that for C++, using classic vtables and interface tables,
rather than the prototype-based runtime model used for JavaScript.

\subsection{Eliminating Types: Any, Union, Intersection}

By default, the TypeScript compiler assigns the Any type to an expression/declaration for which it is unable to
infer a more precise type. STS uses the \emph{--noImplicitAny} option to direct the TypeScript compiler to raise an error
whenever it makes an (implicit) assignment of the Any type.  STS also checks for an explicit use of the Any type in
the program and raises an error. Many JavaScript constructs can only be typed with the Any type, including: prototype lookup,
the \emph{eval} and \emph{with} statements, a \emph{this} reference outside of class context, an index access on non-array object, and
reflection on Function/Object objects. Therefore, these constructs are not in STS.
STS also excludes TypeScript's union and intersection types.

\subsection{Partitioning of Object Types}

In TypeScript, object types are used to describe dictionaries, functions, arrays, class instances,
as well as objects that take on multiple of the above roles, as is common in JavaScript. Object types are
related to one another by structural subtyping.  Interface declarations name object types; classes add implementation
and the ability to statically inherit implementation from super classes. Per the TypeScript specification~\cite{TSspec2016}:
\begin{itemize}
\item ``object types are composed from properties, call signatures, construct signatures, and index signatures, collectively called members'';
\item ``interfaces provide the ability to name and parameterize object types and to compose existing named object types into new ones.''.
\end{itemize}
Object types and the interface declarations that describe them enable the typing of a JavaScript object that plays multiple roles
(as a function, dictionary, etc.). For example, here is an interface that describes an object that is a function from number to number
and also has property (foo) which is a string:
\begin{lstlisting}
interface Bar {
    (x: number): number; // call signature member
    foo: string;         // property member
}
\end{lstlisting}
STS partitions the space of object types as follows:
\begin{itemize}
\item[1.] a \emph{record} type has only (member) properties;
\item[2.] a \emph{function} type has exactly one member, a call signature;
\item[3.] a \emph{constructor function} type: has at least one constructor signature and no other signature kind;
\item[4.] an \emph{array} type has a numeric index signature and no other signatures;
\item[5.] an \emph{other} type: object type not covered by the previous four categories.
\end{itemize}
STS excludes all ``other'' types, leaving us with the four kinds of types.
The partitioning is further reflected in STS's subtype relation:
types $T$ and $U$ from different partitions (in the first four partitions above)
are not related by STS's subtype relation.

\subsection{Treatment of Classes and Methods}

TypeScript introduces two (object) types for each class declaration:
\begin{itemize}
\item a constructor function type by which instances of a class
are created (and static properties accessed), as seen in the previous section;
\item a class type describing the instances of the class, which is a record type.
\end{itemize}
TypeScript's treatment of class types as record types
means that it is possible to assign to a property representing a method,
destroying the class abstraction that ties together code and data.
This also eliminates the possibility of using a shared vtable implementation for classes.

Instead, STS adds a new partition to represent class types, separate from record types,
describing the instances of classes. In STS, a class type is a
record type with an explicit association to its class declaration.
Classic nominal subtyping is applied to class types, with function subtyping used to relate methods with
the same name in subclass and superclass, per TypeScript, just as record types are treated.
Furthermore, STS permits a cast from a class type
to a record supertype, but does not allow any casts from record types to class types. This means
it is possible to use interfaces to abstract over classes, as in Java and C\#.

Finally, STS enforces that a function may not be assigned to a property of a record type,
though it is possible to construct records with function properties using object literals.

\subsection{Subtyping and Type Checking}

Given the above, STS's subtype relation is unsurprising and defined in the appendix.
STS type checking for (pure) expressions remains the same as in TypeScript.
Type checking at assignments, where a value of type $S$ flows (is assigned to)
into a location of type $T$ uses the new subtyping relation: $S$
must be a subtype of $T$.  Downcasts (a supertype being assigned to a subtype)
are not permitted in STS.



