\section{Static TypeScript}
\label{sec:sts}

TypeScript is a typed superset of JavaScript designed to enable JavaScript developers to take advantage of code 
intellisense, static checking and refactoring made possible by types (\url{http://www.typescriptlang.org/}). 
As a starting point, every JavaScript program is a TypeScript program.  Types can be added gradually to such programs, 
supported by type inference. In TypeScript, the Any type 
represents any JavaScript value with no constraints. Type inference may assign the Any type to expressions for which 
no more specific type can be inferred.

While TypeScript provides programming abstractions (classes and interfaces) with syntax like Java/C\#, their semantics
are quite different, as they are based on JavaScript.  Classes are simply syntactic sugar for creating objects that
have code associated with them, but these objects are JavaScript objects with all their dynamic semantics intact. This
is not surprising, as TypeScript was designed to accommodate the dynamic nature of JavaScript and programming patterns
familiar to the JavaScript programmer. 

We approach TypeScript from the viewpoint of the microcontroller programmer, who is 
familiar with the static (though unsound) type systems of C and C++. Our realization is that TypeScript contains a
statically-typed sound subset (Static TypeScript: STS, for short) that closely resembles Java and C\# in its semantics.
STS arises from TypeScript by:
\begin{itemize}
\item \emph{eliminating} the Any type from the type system, as well as JavaScript constructs that only can be typed with Any;
\item \emph{partitioning} TypeScript's space of object types into functions, records, constructor functions, and arrays, with no casts 
    possible between the partitions (and eliminating all other object types) - structural subtyping is used to relate records 
    to each other, as in TypeScript;
\item typing of classes \emph{nominally} rather than structurally, as in TypeScript, and using traditional function (contravariant 
    in the argument type, covariant in the return type) subtyping for functions and methods;
\item \emph{restricting} casts between records and classes: a class can be cast to a record, but a record cannot be cast to a class.
\end{itemize}
STS is a syntactic subset of TypeScript -- there is no change to the syntax of TypeScript.
\emph{[TODO: also, we disallow update of method property]}
In terms of type checking, STS also is a subset of TypeScript. Simply put, given a STS program $P$,
for every type compatibility or subtype relation $R(P)$ of TypeScript, the corresponding relation $R'(P)$ in STS
is a subset of relation $R$. Every STS program is a TypeScript program. 
\emph{[TODO: semantics preserved by erasure for programs written entirely in STS]}
STS guarantees the absence of runtime type errors, including downcasts.
The major class of runtime errors still possible is dereference of null/undefined values. 
STS can be supported by a simple runtime model, like that for C++, using classic vtables and interface tables,
rather than the prototype-based runtime model used for full TypeScript/JavaScript.

\subsection{Eliminating Types: Any, Union, Intersection}

By default, the TypeScript compiler assigns the Any type to an expression/declaration for which it is unable to 
infer a more precise type. STS uses the \emph{--noImplicitAny} option to direct the TypeScript compiler to raise an error 
whenever it makes an (implicit) assignment of the Any type.  STS also checks for an explicit use of the Any type in
the program and raises an error. Many JavaScript constructs can only be typed with the Any type, including: prototype lookup,
the \bf{eval} and \bf{with} statements, a \bf{this} reference outside of class context, an index access on non-array object, and
reflection on Function/Object objects. Therefore, these constructs are not in STS. For good measure,
STS also excludes TypeScript's union and intersection types.

\subsection{Partitioning of Object Types}

In TypeScript, object types are used to describe dictionaries, functions, arrays, as well as class instances. Object 
types also describe objects that take on multiple of the above roles, as is common in JavaScript. Object types are
related to one another by structural subtyping [].  Interface declarations name object types; classes add implementation
and the ability to statically inherit implementation from super classes. Per the TypeScript specification~\cite{XYZ}:
\begin{itemize}
\item ``object types are composed from properties, call signatures, construct signatures, and index signatures, collectively called members'';
\item ``interfaces provide the ability to name and parameterize object types and to compose existing named object types into new ones.''.
\end{itemize} 
Object types and the interface declarations that describe them enable the typing of a JavaScript object that plays multiple roles
(as a function, dictionary, etc.). For example, here is an interface that describes an object that is a function from number to number
and also has property (foo) which is a string:
\begin{lstlisting}
interface Bar {
    (x: number): number; // call signature member
    foo: string;         // property member
}
\end{lstlisting}
STS partitions the space of object types as follows:
\begin{itemize}
\item[1.] a \emph{record} type has only member properties and optionally, a string index signature;
\item[2.] a \emph{function} type has exactly one member, a call signature;
\item[3.] a \emph{constructor function} type: has at least one constructor signature and no other signature kind;
\item[4.] an \emph{array} type has a numeric index signature and no other signatures;
\item[5.] an \emph{other} type: object type not covered by the previous four categories.
\end{itemize}
STS excludes all ``other'' types, leaving us with a set of classic types, as found in Java or C\#.
The partitioning is further reflected in STS' subtype relation, which does cross
partitions. That is types $T$ and $U$ from different partitions above are not related 
by STS' subtype relation. 

\subsection{Nominal Typing of Classes}

To the above four partitions, STS adds a parition to represent class type.

TypeScript introduces two (object) types for each class declaration: a constructor function type by which instances of a class are
created (using new) and static properties accessed; a class type describing the instances of the class (records, as defined above). 
[but the class type contains a link back to the class declaration;  STS treatment of classes using a nominal interpretation, names 
and extends for subtyping]

% Subtyping and type compatibility: program first must pass all the TypeScript checks (to catch errors); we then perform subtyping and type compatibility checks again.  
% “The primitive types are the Number, Boolean, String, Symbol, Void, Null, and Undefined types and all user defined enum types.”
% Disallowed casts.

\subsection{Compilation of STS}

Classic vtable for classes (prefix); classic interface table for classes and records
