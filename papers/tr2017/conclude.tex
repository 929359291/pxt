\section{Conclusion}
\label{sec:conclude}

% \begin{itemize}
%     \item Introducing TS and Blocks to this domain brings a range of immediate, 
%         well understood benefits to users (Ease of programming, memory safety, 
%         removal of syntactic errors, simplified event driven programming models). 
%     \item Our combination of native C++ and TS also allows for low level optimisations 
%           by users if they need it.
%     \item Our software only approach means we have a wide platform base – 
%          we are able to run on a diverse range of small, simple MCUs.
%     \item Native compiled approach also promises performance advantages over 
%           an interpreted bytecode VM approach.
% \end{itemize}

Today, a wide variety of programming languages and tools are available for building smartphone, PC and web apps. Some of these target novices who are learning to code, while others address a variety of professional developers. However, the choice of languages and tools suitable for microcontroller-class devices is much more limited. This is at odds with the tremendous volume of microcontrollers which ship in all manner of embedded devices, and also with the growing interest in learning to code using microcontrollers. 

In this paper we have presented and evaluated a new platform which is designed to address this shortcoming, by bringing modern language features and more modern tooling to microcontrollers. Our aim was to do this in an extensible way which supports novice programmers with block-based programming while providing a progression path to a text-based language for more complex code and ultimately to C++. Our platform includes a new runtime called RT which is designed to make efficient use of the limited resources on a microcontroller, with a particular focus on code size efficiency. A statically-typed subset of TypeScript forms the basis for both blocks- and text-based programs, which are created and compiled using a new web-based IDE which we call WA.

Our aspiration is to enable a new paradigm for programming pretty-much anything, even an Arduino Uno-class MCU, by anyone -- novices and professional developers alike, from anywhere, i.e. without the need for traditional and often quite heavyweight embedded toolchains and IDEs.

% Hardware partners already have started to create \MC packages for the micro:bit.
% Seeed Studio (\url{https://www.seeedstudio.com/}) has created packages to add its Grove components to a micro:bit.
% Grove components are accessed via the I2C serial protocol, supported by the micro:bit device runtime.
% All micro:bit packages for the Grove components are authored in Static TypeScript (gesture, ultrasonic-ranger,
% 4-digital-display, two-led-matrix). These packages can be found under GitHub user ``Tinkertanker'', prefixed with
% ``pxt-''. Sparkfun has created \MC packages for its micro:bit shields (GitHub user ``sparkfun'').

% https://github.com/Tinkertanker/pxt-ssd1306-microbit
% https://github.com/Tinkertanker/pxt-ir-microbit 
% https://github.com/Tinkertanker/pxt-ky040-microbit
% https://github.com/Tinkertanker/pxt-ds1307-microbit
% Sparkfun
% •	https://github.com/sparkfun/pxt-weather-bit
% •	https://github.com/sparkfun/pxt-moto-bit 
% •	https://github.com/sparkfun/pxt-gamer-bit 
% Common packages
% •	https://github.com/Microsoft/pxt-common-packages
% •	Structure:
% o	Libs/package/
% [talk about C++] Various examples.
% [more about customer editor associated with a package]
